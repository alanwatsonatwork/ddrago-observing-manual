\chapter{Observing}

DDRAGO is currently being offered for science observations on best-effort and shared-risk basis, on the understanding that commissioning tasks and the transient science program may take priority, and with the warning that the instrument is not fully characterized. 

The transients team currently observes by taking 60 seconds exposures and dithering randomly in a circle of diameter 1 arcmin. This seems to give a good balance between efficiency, reaching the sky limit, being able to form a sky image, and not losing too much image quality to telescope tracking errors. Other strategies are possible, and one of the aims of this shared-risk time is to allow observers to determine the best strategies for their science and to communicate their requirements and experience to us.

We aim to keep individual observing blocks to about 30 minutes of real time (i.e., 24 x 60 second exposures plus overheads). If more data is required, we recommend repeating blocks.

As a rule of thumb, a typical measured 10-sigma limiting magnitudes for point sources in 60 seconds in the $g$, $r$, and $i$ filters in grey time is $\AB \approx 20$, so in $n$ exposures a typical $m$-sigma limiting magnitude is 
\begin{align}
\AB ≈ 20 + 1.25 \log_{10}(n) − 2.5 \log_{10}(m / 10)
\end{align}
These values obviously vary with seeing and sky brightness. We recommend taking initial test observations before submitting larger requests.
